\documentclass[a4paper,12pt]{article}

\usepackage{amsmath}
%\usepackage{german}
\usepackage{amsfonts}
\usepackage{bbm}
\usepackage{todonotes}

% Setup the page
\setlength{\oddsidemargin}{1in}
\setlength{\evensidemargin}{1in}
\setlength{\topmargin}{1in}

\setlength{\textwidth}{210mm}
\addtolength{\textwidth}{-\oddsidemargin}
\addtolength{\textwidth}{-\evensidemargin}

\setlength{\textheight}{297mm}
\addtolength{\textheight}{-2\topmargin}
\addtolength{\textheight}{-\headheight}
\addtolength{\textheight}{-\headsep}
\addtolength{\textheight}{-\footskip}

\addtolength{\oddsidemargin}{-1in}
\addtolength{\evensidemargin}{-1in}
\addtolength{\topmargin}{-1in}

\setlength{\parindent}{0in}

\pagestyle{empty}

\renewcommand{\labelenumi}{(\alph{enumi})}
\newcommand{\clnote}[1]{\todo[inline,color=blue!20]{CL: #1}}

% Create task enviroment
\newcounter{task}

\newenvironment{task}[2][]
{\goodbreak\def\a{}\def\b{#1}\par\bigskip\stepcounter{task}\textbf{Exercise \thetask{\if\a\b\else\ (#1)\fi}}\hfill\def\b{#2}\if\a\b\else(#2)\fi\nopagebreak\par}
{}

% Define mathematical constants
\newcommand{\setR}{\mathbbm{R}}
\newcommand{\setN}{\mathbbm{N}}
\newcommand{\setP}{\mathbbm{P}}
\newcommand{\setQ}{\mathbbm{Q}}
\newcommand{\setZ}{\mathbbm{Z}}

\renewcommand{\div}{\mathrm{div}}
\newcommand{\grad}{\mathrm{grad}}
\newcommand{\rot}{\mathrm{rot}}

\newcommand{\pprime}{{\prime\prime}}

\newcommand{\conv}{\mathrm{conv}}
\newcommand{\dist}{\mathrm{dist}}
\newcommand{\supp}{\mathrm{supp}}
\renewcommand{\ker}{\mathrm{kern}}
\newcommand{\sgn}{\mathrm{sign}}

\newcommand{\esssup}{\mathop{\mathrm{ess\,sup}}}

% Create draft macro
\newcommand{\false}[2]{#2}
\newcommand{\true}[2]{#1}

\newcommand{\ifdraft}{\false}

% Create head of sheet
\newcommand{\head}[3]{
  XLAB Science Camp August 12th -- 15th 2024   \\
  

\vspace{3\baselineskip}
\begin{center}
Exercise\\[0.25\baselineskip]
{\large\textbf{Numerical simulation of physical systems:}\\[0.25\baselineskip]}
% SS 2016 --- #1
\end{center}

\vspace*{\baselineskip}
\ifdraft{\centerline{--- DRAFT ---}}{\vspace*{\baselineskip}}
\vspace*{\baselineskip}
\def\a{}\def\b{#3}\if\a\b\else{\textbf{Abgabe:} #3, vor der Vorlesung\par}\fi
}


%\renewcommand{\ifdraft}{\true}

\begin{document}
\head{}{}{}
 
\begin{task}[Solution of the Heat Equation]{}

We consider a rod of length $l$, mathematically a one-dimensional domain $\Omega=(0:l)$. The rod has a constant thermal conductivity $k$ and is heated with a constant heat flux density $q_Q$. Therefore, $q_Q>0$.
 
\begin{enumerate}
 \item 
 Calculate the temperature distribution $u(x)$ for the stationary case from the heat equation
 $$
 -k \frac{d^2u}{dx^2}(x) = q_Q
 $$
 \item 
 Show that the temperature distribution in the stationary case can be expressed as
 \begin{equation}\label{eq:tempdistribution}
 u(x)=-\frac{q_Q}{2k} \cdot x^2 + \frac{du}{dx}(0) \cdot x + u(0)
 \end{equation}
 \item
 Specify which expressions in this equation are material parameters and which are boundary conditions (what type of boundary conditions?).
 \item
 Consider now an experiment where both boundary conditions are Dirichlet boundary conditions, where the temperature at the ends of the rod is given by $u(0)$ and $u(l)$. Bring Equation (\ref{eq:tempdistribution}) into a form where only these two boundary conditions appear.
 \item
 Due to the heating of the rod, it is possible that the temperature maximum of the rod is not at the rod ends but in between. Show that a temperature maximum between the rod ends can only occur under the condition
 $$
 \left|u(l)-u(0)\right|<\frac{q_Ql^2}{2k}
 $$
 What does this mean physically?
\end{enumerate}
\end{task}

\end{document}