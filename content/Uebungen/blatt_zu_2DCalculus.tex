\documentclass[a4paper,12pt]{article}

\usepackage{amsmath}
%\usepackage{german}
\usepackage{amsfonts}
\usepackage{bbm}
\usepackage{todonotes}

% Setup the page
\setlength{\oddsidemargin}{1in}
\setlength{\evensidemargin}{1in}
\setlength{\topmargin}{1in}

\setlength{\textwidth}{210mm}
\addtolength{\textwidth}{-\oddsidemargin}
\addtolength{\textwidth}{-\evensidemargin}

\setlength{\textheight}{297mm}
\addtolength{\textheight}{-2\topmargin}
\addtolength{\textheight}{-\headheight}
\addtolength{\textheight}{-\headsep}
\addtolength{\textheight}{-\footskip}

\addtolength{\oddsidemargin}{-1in}
\addtolength{\evensidemargin}{-1in}
\addtolength{\topmargin}{-1in}

\setlength{\parindent}{0in}

\pagestyle{empty}

\renewcommand{\labelenumi}{(\alph{enumi})}
\newcommand{\clnote}[1]{\todo[inline,color=blue!20]{CL: #1}}

% Create task enviroment
\newcounter{task}

\newenvironment{task}[2][]
{\goodbreak\def\a{}\def\b{#1}\par\bigskip\stepcounter{task}\textbf{Exercise \thetask{\if\a\b\else\ (#1)\fi}}\hfill\def\b{#2}\if\a\b\else(#2)\fi\nopagebreak\par}
{}

% Define mathematical constants
\newcommand{\setR}{\mathbbm{R}}
\newcommand{\setN}{\mathbbm{N}}
\newcommand{\setP}{\mathbbm{P}}
\newcommand{\setQ}{\mathbbm{Q}}
\newcommand{\setZ}{\mathbbm{Z}}

\renewcommand{\div}{\mathrm{div}}
\newcommand{\grad}{\mathrm{grad}}
\newcommand{\rot}{\mathrm{rot}}

\newcommand{\pprime}{{\prime\prime}}

\newcommand{\conv}{\mathrm{conv}}
\newcommand{\dist}{\mathrm{dist}}
\newcommand{\supp}{\mathrm{supp}}
\renewcommand{\ker}{\mathrm{kern}}
\newcommand{\sgn}{\mathrm{sign}}

\newcommand{\esssup}{\mathop{\mathrm{ess\,sup}}}

% Create draft macro
\newcommand{\false}[2]{#2}
\newcommand{\true}[2]{#1}

\newcommand{\ifdraft}{\false}

% Create head of sheet
\newcommand{\head}[3]{
  XLAB Science Camp August 12th -- 15th 2024   \\
  

\vspace{3\baselineskip}
\begin{center}
Exercise\\[0.25\baselineskip]
{\large\textbf{Numerical simulation of physical systems:}\\[0.25\baselineskip]}
% SS 2016 --- #1
\end{center}

\vspace*{\baselineskip}
\ifdraft{\centerline{--- DRAFT ---}}{\vspace*{\baselineskip}}
\vspace*{\baselineskip}
\def\a{}\def\b{#3}\if\a\b\else{\textbf{Abgabe:} #3, vor der Vorlesung\par}\fi
}


%\renewcommand{\ifdraft}{\true}

\begin{document}
\head{}{}{}

\begin{task}[Heat Flow]{}
We consider the temperature distribution over the area
$
 \Omega = [0,1 cm]^2 
$, i.e., $ x \in [0,1 cm]$, $ y \in [0,1 cm]$. We recall the heat flow model:
$$
 \vec{q}(x,y) = - k \nabla u(x,y)
$$
A temperature distribution of
$$
 u(x,y) = (1 + \cos(\pi \cdot (x+y))) \cdot 20^\circ C
$$
is measured. The thermal conductivity is $k = 0.625 \frac{W}{K m} $.

\begin{enumerate}
\item Calculate the heat flow across the boundary of $\Omega$. \\
  \textbf{Hint}:
  \begin{itemize}
  \item Consider one boundary piece after another, e.g., $[0,1 cm] \times \{0\}$ (i.e., $(x,0)$ for $x \in [0,1 cm]$) for the lower boundary.
  \item What is the normal direction on this boundary piece? Which partial derivative is, therefore, relevant? Calculate this.
  \item Then, integrate over the boundary piece (only one running variable)
  \item Finally, sum up the flows.
  \end{itemize}
  \item Assuming that the temperature field obeys the heat conduction equation. Can you tell whether the area is being heated or cooled?
\end{enumerate}
\end{task}

\begin{task}[Boundary Conditions]{}
We consider the temperature distribution over the area
$
 \Omega = [0,1 cm]^2 
$, i.e., $ x \in [0,1 cm]$, $ y \in [0,1 cm]$. It applies the partial differential equation
$$
  - k \Delta u (x,y) = f \quad \text{for all} (x,y) \in \Omega, \text{i.e., for all } x \in (0,1) \text{ and } y \in (0,1).
$$
We choose a heat source $f=1$. 
We need boundary conditions to get to a unique solution. For this, we specify the temperature at the left and right borders, $u(\hat{x},y)=20^\circ C$ for $\hat{x} \in \{0,1\}$ and assume that the upper and lower borders are perfectly insulated, i.e., the heat flow to the outside is zero.

\begin{enumerate}
\item Formulate the equations for the boundary conditions.
\item Due to the boundary conditions, we can suspect that the vertical component of the heat flow is zero in the entire region. Determine \emph{one} solution to the problem analytically under this assumption. (Since the problem has exactly one unique solution, this is the only solution.) \\
  \textbf{Remark:} \\
  The calculation of closed solutions is only possible in exceptional cases (like this example). Next, we will use numerical calculations to determine solutions.
\end{enumerate}
\end{task}

\end{document}
