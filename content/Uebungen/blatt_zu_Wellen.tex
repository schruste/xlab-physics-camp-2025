\documentclass[a4paper,12pt]{article}

\usepackage{amsmath}
%\usepackage{german}
\usepackage{amsfonts}
\usepackage{bbm}
\usepackage{todonotes}

% Setup the page
\setlength{\oddsidemargin}{1in}
\setlength{\evensidemargin}{1in}
\setlength{\topmargin}{1in}

\setlength{\textwidth}{210mm}
\addtolength{\textwidth}{-\oddsidemargin}
\addtolength{\textwidth}{-\evensidemargin}

\setlength{\textheight}{297mm}
\addtolength{\textheight}{-2\topmargin}
\addtolength{\textheight}{-\headheight}
\addtolength{\textheight}{-\headsep}
\addtolength{\textheight}{-\footskip}

\addtolength{\oddsidemargin}{-1in}
\addtolength{\evensidemargin}{-1in}
\addtolength{\topmargin}{-1in}

\setlength{\parindent}{0in}

\pagestyle{empty}

\renewcommand{\labelenumi}{(\alph{enumi})}
\newcommand{\clnote}[1]{\todo[inline,color=blue!20]{CL: #1}}

% Create task enviroment
\newcounter{task}

\newenvironment{task}[2][]
{\goodbreak\def\a{}\def\b{#1}\par\bigskip\stepcounter{task}\textbf{Exercise \thetask{\if\a\b\else\ (#1)\fi}}\hfill\def\b{#2}\if\a\b\else(#2)\fi\nopagebreak\par}
{}

% Define mathematical constants
\newcommand{\setR}{\mathbbm{R}}
\newcommand{\setN}{\mathbbm{N}}
\newcommand{\setP}{\mathbbm{P}}
\newcommand{\setQ}{\mathbbm{Q}}
\newcommand{\setZ}{\mathbbm{Z}}

\renewcommand{\div}{\mathrm{div}}
\newcommand{\grad}{\mathrm{grad}}
\newcommand{\rot}{\mathrm{rot}}

\newcommand{\pprime}{{\prime\prime}}

\newcommand{\conv}{\mathrm{conv}}
\newcommand{\dist}{\mathrm{dist}}
\newcommand{\supp}{\mathrm{supp}}
\renewcommand{\ker}{\mathrm{kern}}
\newcommand{\sgn}{\mathrm{sign}}

\newcommand{\esssup}{\mathop{\mathrm{ess\,sup}}}

% Create draft macro
\newcommand{\false}[2]{#2}
\newcommand{\true}[2]{#1}

\newcommand{\ifdraft}{\false}

% Create head of sheet
\newcommand{\head}[3]{
  XLAB Science Camp August 12th -- 15th 2024   \\
  

\vspace{3\baselineskip}
\begin{center}
Exercise\\[0.25\baselineskip]
{\large\textbf{Numerical simulation of physical systems:}\\[0.25\baselineskip]}
% SS 2016 --- #1
\end{center}

\vspace*{\baselineskip}
\ifdraft{\centerline{--- DRAFT ---}}{\vspace*{\baselineskip}}
\vspace*{\baselineskip}
\def\a{}\def\b{#3}\if\a\b\else{\textbf{Abgabe:} #3, vor der Vorlesung\par}\fi
}


%\renewcommand{\ifdraft}{\true}

\begin{document}
\head{}{}{}

%\begin{task}[Wellengleichung in 1D]{}
% Wir betrachten die skalare Wellengleichung (in 1D)
%\begin{equation} \label{eq:wave1d}
% \frac{\partial^2 u}{\partial t^2}(x,t) = c^2 \frac{\partial^2 u}{\partial x^2}(x,t)
%\end{equation}
%\begin{enumerate}
%\item Anfangs sei $u(x,0) = v(x) + w(x)$ für zwei bekannte Funktionen $v(x)$ und $w(x)$. Zeigen Sie, dass $u(x,t) = v(x+c \cdot t) + w(x-c \cdot t)$ die Wellengleichung \eqref{eq:wave1d} löst.\\
%  \textbf{Hinweise:}
%  \begin{itemize}
%  \item Die Funktionen $v(x)$ und $w(x)$ repräsentieren die Lösungsanteile der Wellengleichung welche nach links beziehungsweise nach rechts transportiert werden. $u(x,t)$ ist die Überlagerung dieser beiden Transportvorgänge.
%  \item Die Lösung $u(x,t)$ ist nicht eindeutig. Zum Beispiel wäre auch $u(x,t) = v(x-c \cdot t) + w(x+c \cdot t)$ eine Lösung. Um eine eindeutige Lösung zu erhalten muss man zusätzlich Anfangswerte $\frac{\partial u}{\partial t}(x,0)$ vorschreiben.
%  \end{itemize}
%\item Betrachten wir nun \eqref{eq:wave1d} auf dem Interval $[0,q]$ mit $q > 0$. Als Randbedingungen verlangen wir
%$$
% u(0,t) = \sin(\pi \cdot c \cdot t) \qquad \text{und} \qquad  \frac{\partial u}{\partial t}(q,t) + c \frac{\partial u}{\partial x}(q,t) = 0
% $$
% und als Anfangswerte schreiben wir vor
% $$
% u(x,0) = -\sin(\pi \cdot x) \qquad \text{und} \qquad
%\frac{\partial u}{\partial t}(x,0) = c \pi \cdot  \cos(\pi \cdot x).
%$$
%Geben Sie die Lösung der Wellengleichung an. \\
%\textbf{Hinweise:} \\
%Bedenken Sie eine mögliche Zerlegung wie in Aufgabenteil (a).
%\item Skizzieren Sie die Lösung für $t=0$, $t=\frac{1}{4c}$, $t=\frac{1}{2c}$.
%\item Welche physikalische Bedeutung hat die Randbedingung am rechten Rand $q$?  
%\end{enumerate}
%\end{task}


\begin{task}[Elektromagnetische Schwingungen und Wellen]{}
Elektromagnetische Felder gehorchen den sogenannten Maxwell-Gleichungen, die wir hier für den 3D-Fall angeben wollen. In dieser Aufgabe wollen wir diese relativ komplizierten Gleichung auf eine einfachere \emph{skalare} Wellengleichung überführen.

Sei 
$$
\vec{B}(x,y,z, t) = (B_x(x,y,z, t),B_y(x,y,z, t),B_z(x,y,z, t))
$$ 
die magnetische Flussdichte (Einheit $[B] = 1 T$) und 
$$
\vec{E}(x,y,z, t) = (E_x(x,y,z, t),E_y(x,y,z, t),E_z(x,y,z, t))
$$
die elektrische Feldstärke (Einheit $[E] = 1 V / m$).
\begin{itemize}
  \item Änderung der magnetischen Flussdichte führen zu einem Wirbelfeld:
  \begin{equation} \label{eq:maxw1}
\frac{\partial \vec{B}}{\partial t}(x,y,z, t) = - \left( \begin{array}{c}
 \frac{\partial {E}_z}{\partial y}(x,y,z, t) - \frac{\partial {E}_y}{\partial z}(x,y,z, t) \\
 \frac{\partial {E}_x}{\partial z}(x,y,z, t) - \frac{\partial {E}_z}{\partial x}(x,y,z, t) \\
 \frac{\partial {E}_y}{\partial x}(x,y,z, t) - \frac{\partial {E}_x}{\partial y}(x,y,z, t)
\end{array} \right)
\end{equation}
  \item Elektrische Ströme -- d.h. externe Ströme und Änderungen der elektrischen Feldstärke --  führen zu einem magnetischen Wirbelfeld:
  \begin{equation} \label{eq:maxw2}
  \mu \vec{j} + \mu \varepsilon \frac{\partial \vec{E}}{\partial t}(x,y,z, t) = \left( \begin{array}{c}
 \frac{\partial {B}_z}{\partial y}(x,y,z, t) - \frac{\partial {B}_y}{\partial z}(x,y,z, t) \\
 \frac{\partial {B}_x}{\partial z}(x,y,z, t) - \frac{\partial {B}_z}{\partial x}(x,y,z, t) \\
 \frac{\partial {B}_y}{\partial x}(x,y,z, t) - \frac{\partial {B}_x}{\partial y}(x,y,z, t)
\end{array} \right)
% \mu_0
% \left( \begin{array}{c}
%   j_x \\ j_y \\ j_z
%  \end{array} \right)
\end{equation}
  \item Hierbei ist 
  \begin{itemize}
    \item $\vec{j}$ der Vektor der elektrischen Stromdichte (im folgenden $0$)
    \item $\mu = \mu_0 \mu_r$ die Permittivität ($\mu_r$, relative Permittivität, ein Materialparameter)
    \item $\varepsilon = \varepsilon_0 \varepsilon_r$ ($\varepsilon_r$, relative Dielektrizitätszahl, ein Materialparameter)
    \item $\mu \varepsilon = \frac{1}{c^2}$ wobei $c$ die Lichtgeschwindigkeit im Material ist. \\ Für $\mu_0 \varepsilon_0 = \frac{1}{c_0^2}$ ist $c_0$ die Lichtgeschwindigkeit im Vakuum.
  \end{itemize}
\end{itemize}

\begin{enumerate}
\item Wir nehmen an, dass $\vec{E}$ und $\vec{B}$ unabhängig von der $z$-Koordinate sind, die magnetische Flussdichte in der $x$-$y$-Ebene liegt und das elektrische Feld in $z$-Richtung zeigt, d.h.
\begin{equation} \label{eq:maxw:plane}
\vec{B}(x,y,z, t) = \left( \begin{array}{c} B_x(x,y, t) \\ B_y(x,y, t) \\ 0 \end{array} \right), \qquad 
  \vec{E}(x,y,z, t) = \left( \begin{array}{c} 0 \\ 0 \\ E_z(x,y, t) \end{array} \right). 
\end{equation}  
Leiten Sie hieraus eine skalare Wellengleichung für $E_z(x,y,t)$ her. \\
{\textbf{Hinweis:}} \\
\begin{itemize}
  \item Setzen Sie \eqref{eq:maxw:plane} in \eqref{eq:maxw1} und \eqref{eq:maxw2} ein und leiten sie \eqref{eq:maxw2} nach $t$ ab. 
  \item Es gilt 
  $$
  \frac{ \partial \frac{\partial E_z (x,y,t)}{\partial x}}{\partial t}
  = \frac{\partial^2 E_z (x,y,t)}{\partial x \partial t}
  = \frac{\partial^2 E_z (x,y,t)}{\partial t \partial x} 
  = \frac{ \partial \frac{\partial E_z (x,y,t)}{\partial t}}{\partial x}
  $$
\end{itemize}
\item Wir betrachten die skalare Wellengleichung (in 2D)
\begin{equation} \label{eq:wave}
  \frac{1}{c^2} \frac{\partial^2 u}{\partial t^2}(x,y,t) = \Delta u (x,y,t)
\end{equation}
wobei $c$ die Ausbreitungsgeschwindigkeit der Welle (z.B. Lichtgeschwindigkeit oder Schallgeschwindigkeit) ist. Wenn sich Randbedingungen und Quellen (z.B. $\vec{j}$) in der Zeit nicht ändern, so kann die Lösung zerlegt werden in folgende Anteile:
\begin{equation*}
  u(x,y,t) = \sum_{n=0}^{\infty} \underbrace{\left( c_n \sin(2 \pi n t) + 
  d_n \cos(2 \pi n t) \right)}_{\phi_n(t)} \cdot ~ \hat u_n(x,y) 
\end{equation*}
Wenn Wellen nur für eine spezielle Frequenz angeregt werden, so sind fast alle Koeffizienten $d_n$ und $c_n$ gleich Null und es verbleibt lediglich
\begin{equation*}
  u(x,y,t) = \underbrace{(c_n \sin( \omega t) + d_n \cos( \omega t))}_{\phi_n(t)} \cdot ~ \hat u_n(x,y) , \quad \omega = 2 \pi n.
\end{equation*}
Wir definieren zusätzlich die Wellenzahl $k = \omega / c$ (Einheit $[k] = 1/m$). 
Setzen Sie diesen Ansatz in \eqref{eq:wave} ein. Welcher Differentialgleichung gehorcht $\hat{u}_n(x,y)$?
\item Es sollen die Wellengleichungen für Mikrowellen der Wellenlänge $28mm$ berechnet werden. Da die Wellenlänge bekannt ist, kann der obige Ansatz verwendet werden. Welchen Wert erhalten Sie in diesem Fall für $k$?
\end{enumerate}

\end{task}

\begin{task}[Vorbereitung: Mikrowellen]{}
  Wir betrachten nun 1D Mikrowellen. Am Ort $x=0$ sei ein Mikrowellensender angebracht, der eine Welle der Modulation $u(0,t) = \cos(\omega t)$ von links nach rechts aussendet. Im Interval $[0,a]$ liegt ein Material vor in dem die Lichtgeschwindigkeit $c_1$ ist, während im Interval $[a,b]$ ein Material mit Lichtgeschwindigkeit $c_2$  vorliegt.
  Wir wollen nun die Funktion $u(x,t)$ rekonstruieren, die die Wellengleichung unter diesen Bedingungen löst. Dazu wagen wir folgenden Ansatz:
 \begin{equation} \label{eq:ansatz}
  u(x,t) = \left\{ \begin{array}{cl} \cos(\omega t - k_1 x + \varphi_1)  & \text{ wenn } x \leq a,\\
                                     \cos(\omega t - k_2 x + \varphi_2)  & \text{ sonst, }\end{array} \right.
                                    \end{equation}
                                    wobei
                                    $k_1 = \omega/c_1$ und $k_2 = \omega/c_2$.
                                    Diesem Ansatz liegt die Annahme zugrunde, dass es sich nur um eine von links nach rechts laufende Welle handelt, d.h. es gibt keine Reflexion.
%Wir treffen zusätzlich folgende Annahme:
% $k_1 = 2 \pi M_1 / a$ und  $k_2 = 2 \pi M_2 / a$ für natürliche Zahlen $M_1,~M_2$.
\begin{enumerate}
\item Zeigen Sie dass $u(x,t)$ aus \eqref{eq:ansatz} der Wellengleichung $\frac{\partial^2 u}{\partial t^2}(x,t) = c^2 \frac{\partial^2 u}{\partial x^2}(x,t)
$ gehorcht.
\item Bestimmen Sie nun die Koeffizienten $\varphi_1$, $\varphi_2$, sodass $u(x,t)$ die folgenden zusätzlichen Bedingungen erfüllt:
  \begin{itemize}
  \item Die Welle am Eingang entspricht der eingeprägten Modulation:
    $$u(0,t) = \cos(\omega t)$$
  \item Die Welle ist stetig am Übergang, d.h.:
    $$  \cos(\omega t - k_1 a + \varphi_1) = \cos(\omega t - k_2 a + \varphi_2)$$
  \end{itemize}
\item Zeichnen Sie die Lösung für $t=0$, $t=\pi/4$, $t=\pi/2$ für zwei Fälle:
  \begin{itemize}
    \item $a = 2 \pi$, $b= 4 \pi$, $c_1 = c_2 = \omega = 1$
    \item $a = 2 \pi$, $b= 4 \pi$, $c_1 = \omega = 1$, aber $c_2 = \frac23$.
    \end{itemize}
\item Zeigen Sie, dass die beiden letzten Fälle sich bei $b=4 \pi$ auslöschen.
\end{enumerate}
\end{task}

\end{document}
