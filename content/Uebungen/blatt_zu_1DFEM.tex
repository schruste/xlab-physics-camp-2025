\documentclass[a4paper,12pt]{article}

\usepackage{amsmath}
%\usepackage{german}
\usepackage{amsfonts}
\usepackage{bbm}
\usepackage{todonotes}

% Setup the page
\setlength{\oddsidemargin}{1in}
\setlength{\evensidemargin}{1in}
\setlength{\topmargin}{1in}

\setlength{\textwidth}{210mm}
\addtolength{\textwidth}{-\oddsidemargin}
\addtolength{\textwidth}{-\evensidemargin}

\setlength{\textheight}{297mm}
\addtolength{\textheight}{-2\topmargin}
\addtolength{\textheight}{-\headheight}
\addtolength{\textheight}{-\headsep}
\addtolength{\textheight}{-\footskip}

\addtolength{\oddsidemargin}{-1in}
\addtolength{\evensidemargin}{-1in}
\addtolength{\topmargin}{-1in}

\setlength{\parindent}{0in}

\pagestyle{empty}

\renewcommand{\labelenumi}{(\alph{enumi})}
\newcommand{\clnote}[1]{\todo[inline,color=blue!20]{CL: #1}}

% Create task enviroment
\newcounter{task}

\newenvironment{task}[2][]
{\goodbreak\def\a{}\def\b{#1}\par\bigskip\stepcounter{task}\textbf{Exercise \thetask{\if\a\b\else\ (#1)\fi}}\hfill\def\b{#2}\if\a\b\else(#2)\fi\nopagebreak\par}
{}

% Define mathematical constants
\newcommand{\setR}{\mathbbm{R}}
\newcommand{\setN}{\mathbbm{N}}
\newcommand{\setP}{\mathbbm{P}}
\newcommand{\setQ}{\mathbbm{Q}}
\newcommand{\setZ}{\mathbbm{Z}}

\renewcommand{\div}{\mathrm{div}}
\newcommand{\grad}{\mathrm{grad}}
\newcommand{\rot}{\mathrm{rot}}

\newcommand{\pprime}{{\prime\prime}}

\newcommand{\conv}{\mathrm{conv}}
\newcommand{\dist}{\mathrm{dist}}
\newcommand{\supp}{\mathrm{supp}}
\renewcommand{\ker}{\mathrm{kern}}
\newcommand{\sgn}{\mathrm{sign}}

\newcommand{\esssup}{\mathop{\mathrm{ess\,sup}}}

% Create draft macro
\newcommand{\false}[2]{#2}
\newcommand{\true}[2]{#1}

\newcommand{\ifdraft}{\false}

% Create head of sheet
\newcommand{\head}[3]{
  XLAB Science Camp August 12th -- 15th 2024   \\
  

\vspace{3\baselineskip}
\begin{center}
Exercise\\[0.25\baselineskip]
{\large\textbf{Numerical simulation of physical systems:}\\[0.25\baselineskip]}
% SS 2016 --- #1
\end{center}

\vspace*{\baselineskip}
\ifdraft{\centerline{--- DRAFT ---}}{\vspace*{\baselineskip}}
\vspace*{\baselineskip}
\def\a{}\def\b{#3}\if\a\b\else{\textbf{Abgabe:} #3, vor der Vorlesung\par}\fi
}


%\renewcommand{\ifdraft}{\true}

\begin{document}
\head{}{}{}
 
\begin{task}[Partial Integration / Boundary Conditions]{}
We recall the product rule of differentiation:
$$
 f(x) = g(x) \cdot h(x) \quad \Longrightarrow \quad f'(x) = g'(x) h(x) + g(x) h'(x)
$$
Therefore, with the Fundamental theorem of Calculus, we have
\begin{equation}\label{eq:partInt}
 [f(x)]_a^b = \int_a^b f'(x) dx = \int_a^b \left( g'(x) h(x) + g(x) h'(x) \right) dx = \int_a^b g'(x) h(x) dx + \int_a^b g(x) h'(x) dx.
\end{equation}
 
\begin{enumerate}
  \item 
  Convert the equation
  $$ 
  \int_a^b - k \frac{d^2 u}{d x^2}(x) v(x) \ dx = \int_a^b q_Q(x) v(x) \ dx
  $$
  using \eqref{eq:partInt} into 
  \begin{equation}\label{eq:varform}
      \Longrightarrow  \int_a^b k \frac{d u}{d x}(x) \frac{d v}{d x}(x) \ dx - \Big[k \frac{d u}{dx}(x) v(x) \Big]_a^b = \int_a^b q_Q(x) v(x) \ dx
  \end{equation}
  \item We consider two types of boundary conditions.
  \begin{itemize}
    \item
    "Robin" type boundary conditions, where the (outgoing) heat flux
    at the edge
    $q_S(a) = k \frac{d u}{dx}(a)$ or  $q_S(b) = - k \frac{d u}{dx}(b)$
    is modeled with:
  $$
    q_S(x) = \alpha (u(x) - T^{\text{ref}}), 
  $$
  here $\alpha \in \mathbb{R}$ is the heat transfer coefficient and $T^{\text{ref}} \in \mathbb{R}$ is a reference temperature outside the interval.
    \item
    "Neumann" type boundary conditions, where the heat flux is directly prescribed:
  $$
    q_S(x) = q_S^{\text{ref}} \in \mathbb{R}. 
  $$
  \end{itemize}
  Formulate \eqref{eq:varform} for both cases in such a way that on the left side of the equations all terms with $u$ and derivatives of $u$ stand, and all other terms appear on the right side.
\end{enumerate}
\end{task}

\end{document}